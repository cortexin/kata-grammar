\documentclass[8pt]{book}
\usepackage{gb4e}
\usepackage{multicol}
\usepackage{multirow}
\usepackage{tipa}
\usepackage[inline]{enumitem}
\usepackage[colorlinks]{hyperref}
\usepackage{textcomp}
%Gummi|065|=)


\title{Leyle kata}

\setlength{\oddsidemargin}{0cm}
\setlength{\evensidemargin}{0cm}


\begin{document}


\newcommand{\dictitem}[3]{\item [#1] \textit{#2}. #3}
\newcommand{\dictdef}[2]{\item \textit{#1}. #2}
\newcommand{\inq}{{\color{red} (?)}}
\newcommand{\labial}{\textsuperscript{w}}
\newcommand{\aspirated}{\textsuperscript{h}}
\newcommand{\palatal}{\textsuperscript{j}}
\maketitle


\chapter{Phonology}
%%%%%%%%%%%%
\iffalse
\addtolength{\hoffset}{-32pt}

\begin{center}
\begin{tabular}{|l|l*{10}{|c}|}
\hline
\multicolumn{2}{|l|}{} & labial & dental & alveolar & aveopalatal & palatal & \multicolumn{2}{c|}{velar} & \multicolumn{2}{c|}{uvular} & glottal \\\cline{8-11}

\multicolumn{2}{|l|}{} &&&&&& plain & labial & plain & labial&\\\hline

\multicolumn{2}{|l|}{nasal}  & m  && n && \textltailn & \textipa{N} & \textipa{N}\labial &&&\\\hline

\multirow{4}{*}{stop} & plain &&& t (*d) && k\palatal (*g\palatal) & k (*g) & k\labial (*g\labial)& q (*\textipa{\;G})& q\labial (*\textipa{\;G}\labial) & \textipa{P} \\\cline{2-2}
                      & aspirated &&& t\aspirated (*t) && k\aspirated\palatal (* k\palatal) & k\aspirated (*k) & k\aspirated\labial (*k\labial) & q\aspirated (*q) & q\aspirated\labial (*q\labial) & \\\cline{2-2}
                      & ejective &&& t' && k\aspirated' & k' & k\labial' & q' & q\labial' & \\\hline

\multirow{3}{*}{fricative} & central &&& s & \textipa{S} & x\palatal & x & x\labial & \textipa{X}&\\\cline{2-2}
                           & lateral  &&& \textbeltl &&&&&&\\\hline

\multirow{2}{*}{affricate} & lateral &&& t\textbeltl &&&&& \\\cline{2-2}
                           & siblant &&& ts & t\textipa{S} - d\textipa{Z} &&&& \\\hline

\multirow{2}{*}{ejective} & central &&& ts' & t\textipa{S}' &&&& \\\cline{2-2}
                          & lateral &&& t\textbeltl' &&&&& \\\hline

\end{tabular}
\end{center}
\fi
%%%%%%%%%%%%%%%%%%%%%%%%%%%%%%%%%%%%%%%%%%%%%%%%%%%%%%%%

\section{Inventory}
\begin{center}
\begin{tabular}{|l | l| *{7}{c} |}
	\hline
	\multicolumn{2}{|l|}{} & labial & dental & alveolar & palatal & velar & uvular & glottal \\ \hline
	\multicolumn{2}{|l|}{nasal} & {\color{red} \textsubring{m}} m && {\color{red} \textsubring{n}} n && \textipa{N} && \\\cline{1-2}
	\multirow{2}{*}{stop} & plain  & p &&  t d && k g && \\\cline{2-2}
                          & labialized &&&&& k\textsuperscript{w} g\textsuperscript{w} && \\\cline{1-2}
    \multicolumn{2}{|l|}{fricative}	& f v &
                                   \textipa{T} \textipa{D} &
                                   s z & \textipa{S} \textipa{Z}& x &
                                   {\color{red} \textipa{X}\footnotemark[1]} \textipa{K} &
                                   h {\color{red} \textipa{H}\footnotemark[2]} \\\cline{1-2}
    \multirow{3}{*}{affricate} & siblant &&& ts & {\color{blue} \textsubbar{d}\textipa{Z}\footnotemark[3]} &&&\\\cline{2-2}
                               & non-siblant &&&& \textsubbar{t}\textsubbar{\textipa{\r{\*r}}}. \textsubbar{d}\textsubbar{\textipa{\*r}}. & kx && \\\cline{2-2}
                               & lateral &&& t\textbeltl d\textlyoghlig &&&&\\\cline{1-2}
    \multicolumn{2}{|l|}{approximant} & {\color{red} \textipa{V}\footnotemark[4]} &&& j & w && \\\cline{1-2}
    \multicolumn{2}{|l|}{lateral approximant} && l && \textipa{L} &&& \textltilde  \\\cline{1-2}
    \multicolumn{2}{|l|}{tap} &&& {\color{red} \textipa{R}\footnotemark[5]} &&&& \\\cline{1-2}
	\multicolumn{2}{|l|}{trill} &&& r &&&& \\
	\hline
\end{tabular}
\end{center}
\footnotetext[1]{allophone of [\textipa{K}] after voiceless consonants}
\footnotetext[2]{intervocalic allophone of [h]}
\footnotetext[3]{dialectal variation of [\textipa{Z}]}
\footnotetext[4]{allophone of [v]}
\footnotetext[5]{intervocalic allophone of [d]}

\section{Romanization}
\begin{enumerate}
\item m = [m]
\item mm = [m:]
\item n = [n]
\item nn = [n:]
\item ng = [\textipa{N}] / \#\_
\item rh = [\textipa{X}] / C(-voice)\_V | [\textipa{K}] / otherwise
\item nt = [\textsubring{n}t]
\end{enumerate}

\section{Phonotactics}
\subsection{General rules}
\begin{enumerate}
	\item Syllable structure is (C)(C)V(V)(V)(C)(C).
	\item Word-final consonant clusters receive a murmured schwa at the end (todo examples)
\end{enumerate}


\subsection{Voicing assimilation}
Intervocalic consonants are always voiced, except [s]:
\begin{xlista}
\ex \textit{kiit} [ki:t] - several birds | \textit{kiiten} [ki:\textipa{R}en] - many birds, a lot of birds
\ex \textit{kaisa} [kaisa] - reason
\end{xlista}

Voicing assimilation in consonant clusters is more complicated, but is reflected in the romanization

Stops preceded by a rhotic/nasal/voiced siblant receive +VOICE. Laterals do not participate in this type of assimilation
\begin{exe}
	\ex arda, arhda, anda, azda, alda
	\ex asta, alta
	\ex[*] {anta, arta, azta}
	
\end{exe}

\paragraph{Nasals}

\begin{exe}
\ex V(n,m)CV \textrightarrow n;'[ /ansa/ = [a\textsubring{n}sa], /anza/ = [anza]
\ex VC(n,m)V \textrightarrow /osma/ = [\textopeno s\textsubring{m}a], /ozma/ = [\textopeno zma] 
\end{exe}

\subsection{\textipa{K}}
\paragraph{Historical development} from [q]. Initially only in intervocalic positions, later - everywhere except the syllable coda.

\subsection{s}
\paragraph{Morphonology} When present in a cluster-initial position, devoices the following consonant (except nasals)

\subsection{n}
\paragraph{Morphonology} When present in a cluster-initial position, voices the following consonant (except [s])

\subsection{v}
\paragraph{Related changes}
\begin{enumerate}
\item VbV \textrightarrow V\textipa{B}V
\item b \textrightarrow \textipa{B}
\item \textipa{B} \textrightarrow v
\end{enumerate}



\section{Relative clauses}

\begin{exe}
\ex
\gll roen the lekka nna \\
 man AUX(3SG,) PERF.go away \\
\trans the man who ran away

\ex
\gll roen re leski \\
man AUX(,3SG) PERF.die \\
\trans the man who was killed


\ex
\gll roen nire lekwe eshees \\
man AUX(1SG, 3SG) PERF.give book.INSTR \\
\trans the man to whom i gave a book

\ex
\gll roen thein lekwe eshees \\
man AUX(3SG, 1SG) PERF.give book.INSTR \\
\trans the man who gave me a book

\ex
\gll kay vizo liidval \\
place AUX(2PL,6) APPL-LOC.eat \\
\trans the place where you eat
\end{exe}



\chapter{Verb}

\section{Stem structure}

Prototipical verb stem structure is CV(C)(C)(V) or (C)VC(C)(V). The last (V) is termed \textit{the thematic vowel}. If absent from the stem, the thematic vowel is assumed to be [e].

\subsection{Thematic vowel}
Thematic vowel is ommited from the stem
\begin{itemize}
	\item When followed by certain derivational suffixes
	\item In connected speech transitive constructions
	\item In connected speech 3sg-A constructions
\end{itemize}

\section{Person marking vs Class marking}
All human (and human-like) entities mark agreement on the verb according to their person/number features (3p singular or plural). \\ 
Other entities, when introduced to the discourse, express their agreement with a verbal classifier, which in addition might be used to clarify their meaning (i.e. a single noun can be used with multiple classifiers). Having been introduced to the discourse, the object is then expresses agreement via the 3p.singular pronoun. (TODO: example)

\section{Agent-Patient marking}
Every verb marks its core arguments as either agent-like or patient-like. For person-agreement arguments in the intransitive case:

kiene - to run \\
duala - to be told, to find out passively

\begin{center}
\begin{tabular}{l|l|l|l|l}
\hline
  & Agent & Agent example & Patient & Patient example \\ \hline
1SG & stem+\textbf{i} & kien\textbf{i} & stem+V+\textbf{i/y} & duala\textbf{i} \\ \hline
1PL & stem+V+\textbf{n} & kien\textbf{en} &  stem+V:+\textbf{n} -- stem+V+\textbf{an} & dual\textbf{aan} \\ \hline
2SG & stem+\textbf{es}  & kien\textbf{es} & stem+V:+\textbf{s} -- stem+V+\textbf{is} & duala\textbf{is} \\ \hline
2PL & stem+\textbf{ue} & kien\textbf{ue} & stem+V+\textbf{ve} & duala\textbf{ve} \\ \hline
3 & stem+V & kiene & stem+V+\textbf{re} & duala\textbf{re} \\\hline

\end{tabular}
\end{center}

For class-marking agreement there is no distinction between A and P, and both have the form \textit{stem+V+classifier} (which is effectively patient-like), e.g. \textit{kienella} - it [liquid] runs. \\

With transitive verbs:

\begin{itemize}
	\item If the subject is expressed with a 3p pronoun and the object is expressed with a classifier:
	\begin{itemize}
	\item Agentive subject: stem(+V) \textbf{e}+classifier - \textit{nathe ella} - he drinks it [liquid]
	\item Patientive subject: stem(+V) \textbf(re)+classifier - \textit{sandi rella} - he likes to consume it [liquid] = he likes to drink it 
	\end{itemize}
	\item If the subject is expressed with a 1p pronoun and the object is expressed with a classifier
	\begin{itemize}
	\item Agentive subject: stem(+V) \textbf{i/an}+classifier - \textit{nathe illa / nathe anlla} - I/we drink it [liquid]
	\item Patientive subject: stem(+V) \textbf{ni/dan}+classifier - \textit{raza nidae / raza dandae} - I/we love it [abstract]  
	\end{itemize}
	\item If the subject is expressed with a 2p pronoun and the object is expressed with a classifier
	\begin{itemize}
	\item Agentive subject: stem(+V) \textbf{is/iu}+classifier - \textit{nathe issilla / nathe yulla} - thou/you drink it [liquid]
	\item Patientive subject: stem(+V) \textbf{va/vi}+classifier - \textit{sandi valla / sandi villa} - thou/you like to drink it
	\end{itemize}
	\item If the subject is expressed with a 1p pronoun and the object is expressed with a 2p pronoun - same as with classifiers: 
		\textit{maar iva} - (1sgA-2sg) I make you weep,
		\textit{maar ivi} - (1sgA-2pl) I make you weep,
		\textit{raza niva} - (1sgP-2sg) I love you,
		\textit{raza nivi} - (1sgP-2pl) I love you,
		\textit{maar anva} - (1plA-2sg) We make you weep,
		\textit{maar anvi} - (1plA-2pl) We make you weep,
		\textit{raza danva} - (1plP-2sg) We love you,
		\textit{raza danvi} - (1plP-2pl) We love you. etc
		
	\item 3p pronoun / classifier subject with 1p/2p object is not allowed (TODO examples) 

\end{itemize}

\section{Aux}

\subsection{Historical development of pronoun markers}
\begin{enumerate}
	\item Third person singular A-pronoun is ommitted (not marked overtly on the verb). Third person singular P-pronoun (\textit{de}) is suffixed onto the verb.
	\item The suffixation of \textit{de} causes a shift in accent, to which new speakers pay more attention. Given that most verb stems end in [e], this turns the 3sg-P ending [e\textipa{R}e] into [e:]
	\item By analogy, other persons get their P-pronoun initial consonant deleted
	\item In order to distinguish A-pronouns, most of which begin with a vowel, from P-pronouns, which lost their initial consonant, A-pronouns start eclipsing the final vowel of the verb stem
\end{enumerate}

\subsection{Omission}
Omission of Aux is allowed in the following cases
\begin{enumerate}
	\item Noun incorporation of O. Aux is replaced by an S-agreement suffix on the verb
	\item VSO order with relatively light S and O constituents
	\item with verbs that have suppletion wrt O noun class.
\end{enumerate}


\section{Noun incorporation}



\subsection{Light verbs (and verb forms)}
There is a closed class of verbs that are obligatorily incorporated, that is, they may not appear as free words without an incorporated noun. It is possible to analyze these as verbalizing affixes, but in this P we shall stick to Mithun's terminology and call these verbs \textit{light verbs}, and the verbs that may appear as free words - \textit{heavy verbs}.\\



Generally, light verbs tend to denote more abstract or generic actions than their heavy synonyms:

\begin{tabular}{l r}

\textbf{Light - hyperonym} & \textbf{Heavy - hyponym} \\\\

\textit{X-iys} - to consume X & \textit{vol} - to eat \\
                              & \textit{nath} - to drink \\\\
                              

\textit{X-ezga} - to strike with X & \textit{andula} - to cut \\
                                   & \textit{vedl} - to pierce \\\\
\end{tabular}


TODO - counterexamples, light verbs with specific meaning


\section{Aspect}
\subsection{Aktionsart}

Every (heavy) verb falls into one of 4 classes roughly corresponding to Vendler's typology. The distinction is made on the basis of the following language-internal criteria

\begin{itemize}
	\item Compatibility with the progressive marking - accomplishments and achievements have no progressive form
	\item Semantic tense of the unmarked form - only activities have the present semantics by default
	\item TODO
\end{itemize}

\paragraph{Accomplishments} 
denote actions and events with no internal duration and a clear endpoint, e.g. \textit{rhmis} - to decide,...



\begin{center}
\begin{tabular}{|l|c|c|c|}
\hline
& inherent aspect & progressive & conative \\ \hline
achievements & perfective & no & yes \\ \hline
states & generic & no & no \\\hline
\end{tabular}
\end{center}


\section{Mood}
\subsection{Irrealis}
Unrealized events, obligations, conditions, negative commands. Scoping - irrealis check before or after +neg +q?

\section{Aspect}

\section{Aux and class agreement}

\paragraph{Transitivity}
In order to utilize a free Aux particle, the action must be highly transitive and have Agent-Patient as its core thematic roles. Semantically transitive verbs that do not satisfy these conditions are cast into intransitive forms by various means, among which we list the following

\begin{itemize}
\item \textbf{Noun incorporation}. Habitual activities (a)
1  
  \begin{xlista}
    \ex rhisenuari \\
        rhise-EPEN-varre-1SG \\
        wood-cut-I \\
        i cut wood / i wood-cut
  \end{xlista}

\end{itemize}

\paragraph{Free Aux with intransitives}
Free aux particle may be encountered with a syntactically intransitive verb in the following cases

\begin{itemize}
\item Some light verbs, e.g. \textit{-sayu} 
\item Verb movement to the right
\item Relative constructions
\end{itemize}


\chapter{Noun}

\section{Noun classes}
\subsection{Verbal noun classifiers}

\begin{enumerate}
\item \textit{ta} - long (and relatively small) objects, wooden objects, instruments
\item \textit{lla} - liquids, active events and processes
\item \textit{mwe} - unbounded surfaces, unseen objects
\item \textit{se} - rock,  large round objects
\item \textit{ish} - animals, low rank, pej
\end{enumerate}


\section{Posession}

\subsection{Posessive prefix harmony}

todo

\subsection{Obligatory posession}

Nouns that are inalienably possessed must bear an indeterminate affix \textbf{i(h)-}

\begin{exe}
\ex \textit{naani} - my mother \\
 \textit{ihaani} - (someone's) mother

\ex \textit{nasande} - my liver \\
 \textit{isande} - (someone's) liver

\end{exe}


\section{Noun plurality}
Only nouns denoting sentient beings have obligatory plurality marking

\subsection{Singulative}
TODO

\subsection{Ablaut plural}
Some (usually monosyllabic) nouns form their plural by raising and lengthening their last vowel.



\section{Noun incorporation}

\subsection{Animacy}
Type 3 incorporation is productive in case where P is a non-sentient common noun and A is a sentient denoted by a pronoun (person/number phi agreement). In these cases S-Aux part may be replaced by a verbal affix:

\begin{tabular}{c | c}
  1SG & -i/-y \\
  1PL & -n/-an \\
  2SG & -s/-is \\
  2PL & -ve/-eve \\
  3SG & -re/-e \\
  3PL & -m \\
\end{tabular}

\begin{exe}
\ex 
\gll \textit{ketezrhay (zrha niye ket)} \\
  parrot:see:1SG (see AUX(1SG,3) parrot) \\
\trans i see a parrot

\ex 
\gll \textit{laccovolere (levole theue acco)} \\
  PERF:meat:eat:3SG (PERF:eat AUX(3SG,9) meat) \\
\trans he ate meat
\end{exe}

\subsection{Quantifiers}

\begin{center}
\begin{tabular}{c | c | c}
affix & meaning & example \\ \hline
hw & nothing/noone & \textit{hwraza} - to love noone/nothing \\
\end{tabular}
\end{center}


\section{Morphology}
\subsection{affixes}
\paragraph{Manner affixes}
\begin{center}
\begin{tabular}{| c | c | c |}
  \hline
  \textit{sma-} & sma-garra & kill with bare hands \\ \hline
\end{tabular}
\end{center}

\paragraph{Location-directional affixes} Do we need them?

\chapter{Dictionary}

\section{Abbreviations}
\begin{itemize}
	\item LV - light verb
	\item OA - agentive object 
	\item OP - patientive object
	\item OPl - obligatory plurality
	\item OPoss - obligatory possession
	\item SA - agentive subject
	\item SP - patientive subject
\end{itemize}


\section{Proto-stems}
\begin{multicols}{2}
	\begin{description}
		\dictitem{kay}{}{land, ground, area}
		\dictitem{om/on}{}{mind}
	\end{description}
\end{multicols}


\section{Main}
\begin{multicols}{2}
  \begin{description}
  \dictitem{aanu}{N, cl4, O-Poss}{ mother}
  \item [inha]
  	\begin{enumerate*}
  		\dictdef{N, cl3, PL - innan}{ winter}
  		\dictdef{V} {to spend winter}
  	\end{enumerate*}
  
  \dictitem{ket}{N, cl9, PL - kiit}{ a small bird, usually passerine}
  \dictitem{kissa}{N, cl2}{ need (state of)}
  \dictitem{llel}{V, cl2}{ existential verb}
  \dictitem{-sayu}{LV}{to call X by name, mention X}
  \dictitem{-sun}{Adj-Aff}{ good, well-behaved}
  \dictitem{-s\"ud}{}{fall}
  \dictitem{yon}{V, SA, OP, cl3}{ hunt. also NI with the type of animal being hunted: \textit{ketyon} - to hunt birds, bird-hunt}
  \end{description}
\end{multicols}

\end{document}
