\documentclass[8pt]{book}
\usepackage{gb4e}
\usepackage{multicol}
\usepackage{multirow}
\usepackage{tipa}
\usepackage[inline]{enumitem}
\usepackage[colorlinks]{hyperref}
\usepackage{textcomp}
%Gummi|065|=)


\title{Leyle kata}

\setlength{\oddsidemargin}{0cm}
\setlength{\evensidemargin}{0cm}


\begin{document}


\newcommand{\dictitem}[3]{\item [#1] \textit{#2}. #3}
\newcommand{\dictdef}[2]{\item \textit{#1}. #2}
\newcommand{\inq}{{\color{red} (?)}}
\newcommand{\labial}{\textsuperscript{w}}
\newcommand{\aspirated}{\textsuperscript{h}}
\newcommand{\palatal}{\textsuperscript{j}}
\maketitle


\chapter{Phonology}
%%%%%%%%%%%%
\iffalse
\addtolength{\hoffset}{-32pt}

\begin{center}
\begin{tabular}{|l|l*{10}{|c}|}
\hline
\multicolumn{2}{|l|}{} & labial & dental & alveolar & aveopalatal & palatal & \multicolumn{2}{c|}{velar} & \multicolumn{2}{c|}{uvular} & glottal \\\cline{8-11}

\multicolumn{2}{|l|}{} &&&&&& plain & labial & plain & labial&\\\hline

\multicolumn{2}{|l|}{nasal}  & m  && n && \textltailn & \textipa{N} & \textipa{N}\labial &&&\\\hline

\multirow{4}{*}{stop} & plain &&& t (*d) && k\palatal (*g\palatal) & k (*g) & k\labial (*g\labial)& q (*\textipa{\;G})& q\labial (*\textipa{\;G}\labial) & \textipa{P} \\\cline{2-2}
                      & aspirated &&& t\aspirated (*t) && k\aspirated\palatal (* k\palatal) & k\aspirated (*k) & k\aspirated\labial (*k\labial) & q\aspirated (*q) & q\aspirated\labial (*q\labial) & \\\cline{2-2}
                      & ejective &&& t' && k\aspirated' & k' & k\labial' & q' & q\labial' & \\\hline

\multirow{3}{*}{fricative} & central &&& s & \textipa{S} & x\palatal & x & x\labial & \textipa{X}&\\\cline{2-2}
                           & lateral  &&& \textbeltl &&&&&&\\\hline

\multirow{2}{*}{affricate} & lateral &&& t\textbeltl &&&&& \\\cline{2-2}
                           & siblant &&& ts & t\textipa{S} - d\textipa{Z} &&&& \\\hline

\multirow{2}{*}{ejective} & central &&& ts' & t\textipa{S}' &&&& \\\cline{2-2}
                          & lateral &&& t\textbeltl' &&&&& \\\hline

\end{tabular}
\end{center}
\fi
%%%%%%%%%%%%%%%%%%%%%%%%%%%%%%%%%%%%%%%%%%%%%%%%%%%%%%%%

\section{Inventory}
\begin{center}
\begin{tabular}{|l | l| *{7}{c} |}
	\hline
	\multicolumn{2}{|l|}{} & labial & dental & alveolar & palatal & velar & uvular & glottal \\ \hline
	\multicolumn{2}{|l|}{nasal} & {\color{red} \textsubring{m}} m && {\color{red} \textsubring{n}} n && \textipa{N} && \\\cline{1-2}
	\multirow{2}{*}{stop} & plain  & p &&  t d && k g && \\\cline{2-2}
                          & labialized &&&&& k\textsuperscript{w} g\textsuperscript{w} && \\\cline{1-2}
    \multicolumn{2}{|l|}{fricative}	& f v &
                                   \textipa{T} \textipa{D} &
                                   s z & \textipa{S} \textipa{Z}& x &
                                   {\color{red} \textipa{X}\footnotemark[1]} \textipa{K} &
                                   h {\color{red} \textipa{H}\footnotemark[2]} \\\cline{1-2}
    \multirow{3}{*}{affricate} & siblant &&& ts & {\color{blue} \textsubbar{d}\textipa{Z}\footnotemark[3]} &&&\\\cline{2-2}
                               & non-siblant &&&& \textsubbar{t}\textsubbar{\textipa{\r{\*r}}}. \textsubbar{d}\textsubbar{\textipa{\*r}}. & kx && \\\cline{2-2}
                               & lateral &&& t\textbeltl d\textlyoghlig &&&&\\\cline{1-2}
    \multicolumn{2}{|l|}{approximant} & {\color{red} \textipa{V}\footnotemark[4]} &&& j & w && \\\cline{1-2}
    \multicolumn{2}{|l|}{lateral approximant} && l && \textipa{L} &&& \textltilde  \\\cline{1-2}
    \multicolumn{2}{|l|}{tap} &&& {\color{red} \textipa{R}\footnotemark[5]} &&&& \\\cline{1-2}
	\multicolumn{2}{|l|}{trill} &&& r &&&& \\
	\hline
\end{tabular}
\end{center}
\footnotetext[1]{allophone of [\textipa{K}] after voiceless consonants}
\footnotetext[2]{intervocalic allophone of [h]}
\footnotetext[3]{dialectal variation of [\textipa{Z}]}
\footnotetext[4]{allophone of [v]}
\footnotetext[5]{intervocalic allophone of [d]}

\section{Romanization}
\begin{enumerate}
\item m = [m]
\item mm = [m:]
\item n = [n]
\item nn = [n:]
\item ng = [\textipa{N}] / \#\_
\item rh = [\textipa{X}] / C(-voice)\_V | [\textipa{K}] / otherwise
\item nt = [\textsubring{n}t]
\end{enumerate}

\section{Phonotactics}
\subsection{General rules}
\begin{enumerate}
	\item Syllable structure is (C)(C)V(V)(V)(C)(C).
	\item Word-final consonant clusters receive a murmured schwa at the end (todo examples)
\end{enumerate}


\subsection{Voicing assimilation}
Intervocalic consonants are always voiced, except [s]:
\begin{xlista}
\ex \textit{kiit} [ki:t] - several birds | \textit{kiiten} [ki:\textipa{R}en] - many birds, a lot of birds
\ex \textit{kaisa} [kaisa] - reason
\end{xlista}

Voicing assimilation in consonant clusters is more complicated, but is reflected in the romanization

Stops preceded by a rhotic/nasal/voiced siblant receive +VOICE. Laterals do not participate in this type of assimilation
\begin{exe}
	\ex arda, arhda, anda, azda, alda
	\ex asta, alta
	\ex[*] {anta, arta, azta}
	
\end{exe}

\paragraph{Nasals}

\begin{exe}
\ex V(n,m)CV \textrightarrow n;'[ /ansa/ = [a\textsubring{n}sa], /anza/ = [anza]
\ex VC(n,m)V \textrightarrow /osma/ = [\textopeno s\textsubring{m}a], /ozma/ = [\textopeno zma] 
\end{exe}

\subsection{\textipa{K}}
\paragraph{Historical development} from [q]. Initially only in intervocalic positions, later - everywhere except the syllable coda.

\subsection{s}
\paragraph{Morphonology} When present in a cluster-initial position, devoices the following consonant (except nasals)

\subsection{n}
\paragraph{Morphonology} When present in a cluster-initial position, voices the following consonant (except [s])

\subsection{v}
\paragraph{Related changes}
\begin{enumerate}
\item VbV \textrightarrow V\textipa{B}V
\item b \textrightarrow \textipa{B}
\item \textipa{B} \textrightarrow v
\end{enumerate}



\section{Relative clauses}

\begin{exe}
\ex
\gll roen the lekka nna \\
 man AUX(3SG,) PERF.go away \\
\trans the man who ran away

\ex
\gll roen re leski \\
man AUX(,3SG) PERF.die \\
\trans the man who was killed


\ex
\gll roen nire lekwe eshees \\
man AUX(1SG, 3SG) PERF.give book.INSTR \\
\trans the man to whom i gave a book

\ex
\gll roen thein lekwe eshees \\
man AUX(3SG, 1SG) PERF.give book.INSTR \\
\trans the man who gave me a book

\ex
\gll kay vizo liidval \\
place AUX(2PL,6) APPL-LOC.eat \\
\trans the place where you eat
\end{exe}



\chapter{Verb}

\section{Aux}

\subsection{Omission}
Omission of Aux is allowed in the following cases
\begin{enumerate}
	\item Noun incorporation of O. Aux is replaced by an S-agreement suffix on the verb
	\item VSO order with relatively light S and O constituents
	\item with verbs that have suppletion wrt O noun class.
\end{enumerate}

\subsection{Aux patterns V1}
\paragraph{Class-class aux}

\begin{center}
	\begin{tabular}{|l|c|c|c|c|c|c|c|}
	\hline
	& 1 & 2 & 3 & 4 & 5 & 6 & 9 \\ \hline
	1 & taa & (dath)? & tay & tae & taashi & taaso & taue \\ \hline
	2 & dath(?) & daa & day & dae & daashi & daaso & daue \\ \hline
	3 & yetaa & yedaa & yye & yen & yeshi & yeso & yeue \\ \hline
	4 & anda & anda & aniye & ann & anshi & anzo & angwe \\ \hline
	5 & shita & shida & shiye & shian & shii & shiso & shigwe \\ \hline
    6 & sooth & soodh & soy & san & sooshi & soo & soue \\ \hline
    9 & gweta & gweda & eye(?) & gwean & gweshi & gwezo & gwee \\ \hline
	\end{tabular}
\end{center}

\paragraph{Phi-class aux}
\begin{center}
	\begin{tabular}{|l|c|c|c|c|c|c|c|}
	\hline
	& 1 & 2 & 3 & 4 & 5 & 6 & 9 \\ \hline
    1SG & niita & niida & niye & nian & niishi & nizo & niue  \\ \hline
    1PL & duath & duadh & duye & duan & dushii & duzo & duue \\ \hline 
    2SG & vaath & vaadh & vay & vaan & vashii & vazo & vaue \\ \hline
    2PL & viath & viadh & viye & vian & vishii & vizo & viue \\ \hline
  	3SG & theat & thead & theye & thean & theshii & theso & theue \\ \hline
  	3PL & treat & dread & treye & trean & treshii & treso & tregwe \\ \hline
	\end{tabular}
\end{center}

\paragraph{Class-phi}
\begin{center}
	\begin{tabular}{|l|c|c|c|c|c|c|}
	  \hline
      & 1SG & 1PL & 2SG & 2PL & 3SG & 3PL \\ \hline
      1 & taain & talen & tais & tave & tare & tamma  \\ \hline
      2 & daain & dalen & dais & dave & dare & damma \\ \hline
      3 & yin  & yelen & yis & yeve & yere & yema \\ \hline
      4 & ain & anlen & annis & anue & arre & anna \\ \hline
      5 & shiin & shilen & shiis & shive & shire  & shiima\\ \hline
      6 & soin & zolen & soois & zove & zore & soma \\ \hline
      9 & gwein & gwelen & gweis & gwee??? & gwere & gwema\\ \hline
      \end{tabular}
\end{center}

\section{Mood}
\subsection{Irrealis}
Unrealized events, obligations, conditions, negative commands. Scoping - irrealis check before or after +neg +q?

\section{Aspect}

\section{Aux and class agreement}

\paragraph{Transitivity}
In order to utilize a free Aux particle, the action must be highly transitive and have Agent-Patient as its core thematic roles. Semantically transitive verbs that do not satisfy these conditions are cast into intransitive forms by various means, among which we list the following

\begin{itemize}
\item \textbf{Noun incorporation}. Habitual activities (a)
1  
  \begin{xlista}
    \ex rhisenuari \\
        rhise-EPEN-varre-1SG \\
        wood-cut-I \\
        i cut wood / i wood-cut
  \end{xlista}

\end{itemize}

\paragraph{Free Aux with intransitives}
Free aux particle may be encountered with a syntactically intransitive verb in the following cases

\begin{itemize}
\item Some light verbs, e.g. \textit{-sayu} 
\item Verb movement to the right
\item Relative constructions
\end{itemize}


\chapter{Noun}
\section{Posession}

\subsection{Posessive prefix harmony}

todo

\subsection{Obligatory posession}

Nouns that are inalienably possessed must bear an indeterminate affix \textbf{i(h)-}

\begin{exe}
\ex \textit{naani} - my mother \\
 \textit{ihaani} - (someone's) mother

\ex \textit{nasande} - my liver \\
 \textit{isande} - (someone's) liver

\end{exe}


\section{Noun plurality}
\subsection{Singulative}
TODO

\subsection{Ablaut plural}
Some (usually monosyllabic) nouns form their plural by raising and lengthening their last vowel.



\section{Noun incorporation}

\subsection{Animacy}
Type 3 incorporation is productive in case where P is a non-sentient common noun and A is a sentient denoted by a pronoun (person/number phi agreement). In these cases S-Aux part may be replaced by a verbal affix:

\begin{tabular}{c | c}
  1SG & -i/-y \\
  1PL & -n/-an \\
  2SG & -s/-is \\
  2PL & -ve/-eve \\
  3SG & -re/-e \\
  3PL & -m \\
\end{tabular}

\begin{exe}
\ex 
\gll \textit{ketezrhay (zrha niye ket)} \\
  parrot:see:1SG (see AUX(1SG,3) parrot) \\
\trans i see a parrot

\ex 
\gll \textit{laccovolere (levole theue acco)} \\
  PERF:meat:eat:3SG (PERF:eat AUX(3SG,9) meat) \\
\trans he ate meat
\end{exe}

\subsection{Quantifiers}

\begin{center}
\begin{tabular}{c | c | c}
affix & meaning & example \\ \hline
hw & nothing/noone & \textit{hwraza} - to love noone/nothing \\
\end{tabular}
\end{center}


\section{Morphology}
\subsection{affixes}
\paragraph{Manner affixes}
\begin{center}
\begin{tabular}{| c | c | c |}
  \hline
  \textit{sma-} & sma-garra & kill with bare hands \\ \hline
\end{tabular}
\end{center}

\paragraph{Location-directional affixes} Do we need them?

\section{Dictionary}
\begin{multicols}{2}
  \begin{description}
  \dictitem{aanu}{N, cl4, O-Poss}{ mother}
  \item [inha]
  	\begin{enumerate*}
  		\dictdef{N, cl3, PL - innan}{ winter}
  		\dictdef{V} {to spend winter}
  	\end{enumerate*}
  
  \dictitem{ket}{N, cl9, PL - kiit}{ a small bird, usually passerine}
  \dictitem{kissa}{N, cl2}{ need (state of)}
  \dictitem{llel}{V, cl2}{ existential verb}
  \dictitem{-sayu}{LV}{to call X by name, mention X}
  \dictitem{-sun}{Adj-Aff}{ good, well-behaved}
  \end{description}
\end{multicols}

\end{document}
