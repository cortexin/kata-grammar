\documentclass[11pt]{article}
\usepackage{gb4e}
\usepackage{tipa}
\usepackage{textcomp}
%Gummi|065|=)


\title{Leyle kata}

\setlength{\oddsidemargin}{0cm}
\setlength{\evensidemargin}{0cm}


\begin{document}

\newcommand{\auxthreesg}{the }
\maketitle


\section{Phonology}

\subsection{Voicing assimilation}

\paragraph{Nasals}

\begin{exe}
\ex V(n,m)CV \textrightarrow /ansa/ = [a\textsubring{n}sa], /anza/ = [anza]
\ex VC(n,m)V \textrightarrow /osma/ = [\textopeno s\textsubring{m}a], /ozma/ = [\textopeno zma] 
\end{exe}

\subsection{\textipa{K}}
\paragraph{Historical development} from [q]. Initially only in intervocalic positions, later - everywhere except the syllable coda.

\subsection{s}
\paragraph{Morphonology} When present in a cluster-initial position, devoices the following consonant (except nasals)

\subsection{n}
\paragraph{Morphonology} When present in a cluster-initial position, voices the following consonant (except [s])

\subsection{v}
\paragraph{Related changes}
\begin{enumerate}
\item VbV \textrightarrow V\textipa{B}V
\item b \textrightarrow \textipa{B}
\item \textipa{B} \textrightarrow v
\end{enumerate}

\section{Aux patterns V1}
\subsection{Class-class aux}

\begin{center}
	\begin{tabular}{|l|c|c|c|c|c|c|c|}
	\hline
	& 1 & 2 & 3 & 4 & 5 & 6 & 9 \\ \hline
	1 & taa & (dath)? & tay & tae & taashi & taaso & taue \\ \hline
	2 & dath(?) & daa & day & dae & daashi & daaso & daue \\ \hline
	3 & yetaa & yedaa & yye & yen & yeshi & yeso & yeue \\ \hline
	4 & anda & anda & aniye & ann & anshi & anzo & angwe \\ \hline
	5 & shita & shida & shiye & shian & shii & shiso & shigwe \\ \hline
    6 & sooth & soodh & soy & san & sooshi & soo & soue \\ \hline
    9 & gweta & gweda & eye(?) & gwean & gweshi & gwezo & gwee \\ \hline
	\end{tabular}
\end{center}

\subsection{Phi-class aux}
\begin{center}
	\begin{tabular}{|l|c|c|c|c|c|c|c|}
	\hline
	& 1 & 2 & 3 & 4 & 5 & 6 & 9 \\ \hline
    1SG & niita & niida & niye & nian & niishi & nizo & niue  \\ \hline
    1PL & duath & duadh & duye & duan & dushii & duzo & duue \\ \hline 
    2SG & vaath & vaadh & vay & vaan & vashii & vazo & vaue \\ \hline
    2PL & viath & viadh & viye & vian & vishii & vizo & viue \\ \hline
  	3SG & theat & thead & theye & thean & theshii & theso & theue \\ \hline
  	3PL & treat & dread & treye & trean & treshii & treso & tregwe \\ \hline
	\end{tabular}
\end{center}

\subsection{Class-phi}
\begin{center}
	\begin{tabular}{|l|c|c|c|c|c|c|}
	  \hline
      & 1SG & 1PL & 2SG & 2PL & 3SG & 3PL \\ \hline
      1 & taain & talen & tais & tave & tare & tamma  \\ \hline
      2 & daain & dalen & dais & dave & dare & damma \\ \hline
      3 & yin  & yelen & yis & yeve & yere & yema \\ \hline
      4 & ain & anlen & annis & anue & arre & anna \\ \hline
      5 & shiin & shilen & shiis & shive & shire  & shiima\\ \hline
      6 & soin & zolen & soois & zove & zore & soma \\ \hline
      9 & gwein & gwelen & gweis & gwee??? & gwere & gwema\\ \hline
      \end{tabular}
\end{center}

\section{Relative clauses}

\begin{exe}
\ex
\gll roen \auxthreesg lekka nna \\
 man AUX(3SG,) PERF.go away \\
\trans the man who ran away

\ex
\gll roen re leski \\
man AUX(,3SG) PERF.die \\
\trans the man who was killed


\ex
\gll roen nire lekwe eshees \\
man AUX(1SG, 3SG) PERF.give book.INSTR \\
\trans the man to whom i gave a book

\ex
\gll roen thein lekwe eshees \\
man AUX(3SG, 1SG) PERF.give book.INSTR \\
\trans the man who gave me a book

\ex
\gll kay vizo liidval \\
place AUX(2PL,6) APPL-LOC.eat \\
\trans the place where you eat
\end{exe}


\section{Posession}

\subsection{Posessive prefix harmony}

todo

\subsection{Obligatory posession}

Nouns that are inalienably possessed must bear an indeterminate affix \textbf{i(h)-}

\begin{exe}
\ex \textit{naani} - my mother \\
 \textit{ihaani} - (someone's) mother

\ex \textit{nasande} - my liver \\
 \textit{isande} - (someone's) liver

\end{exe}


\section{Noun incorporation}

\subsection{Animacy}
Type 3 incorporation is productive in case where P is a non-sentient common noun and A is a sentient denoted by a pronoun (person/number phi agreement). In these cases S-Aux part may be replaced by a verbal affix:

\begin{tabular}{c | c}
  1SG & -i/-y \\
  1PL & -n/-an \\
  2SG & -s/-is \\
  2PL & -ve/-eve \\
  3SG & -re/-e \\
  3PL & -m \\
\end{tabular}

\begin{exe}
\ex 
\gll \textit{ketezrhay (zrha niye ket)} \\
  parrot:see:1SG (see AUX(1SG,3) parrot) \\
\trans i see a parrot

\ex 
\gll \textit{laccovolere (levole theue acco)} \\
  PERF:meat:eat:3SG (PERF:eat AUX(3SG,9) meat) \\
\trans he ate meat
\end{exe}

\subsection{Quantifiers}

\begin{center}
\begin{tabular}{c | c | c}
affix & meaning & example \\ \hline
hw & nothing/noone & \textit{hwraza} - to love noone/nothing \\
\end{tabular}
\end{center}

\end{document}
